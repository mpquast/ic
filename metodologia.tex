\documentclass[12pt, a4paper]{article}
\usepackage[utf8]{inputenc}
\usepackage{indentfirst}


	\begin{document}

	\title{Metodologia}
	\author{Yasmine Brasco}
	\date{25/09/2016}

Segundo Charnet, Freire, Charnet e Bonvino (1999, p. 11), dado Y uma variável resposta e X uma variável preditora:

%falta fazer o recuo

O modelo de regressão linear simples descreve a variável Y como uma soma de uma quantidade determinística e uma quantidade aleatória. A parte determinística, uma reta em função de X, representa a informação sobre Y que já poderíamos ``esperar'', apenas com o conhecimento da variável X.

%termino do recuo



\end{document}